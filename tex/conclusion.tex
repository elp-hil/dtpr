%% conclusion.tex
%%

%% ==================
\chapter{Conclusion \& Outlook}
\label{ch:Conclusion}
KATRIN is moving forward to finding the neutrino mass - and another part contributing to the whole experiment has been completed with the muon detection system operational and taking data. 
At the main spectrometer, set up has been completed. The monitor spectrometer system was readopted. Both systems are able to take data at rates that compare well to literature values and simulations.
Many settings had to be adjusted for the detection system to realize its full potential. High voltage supplies were installed, software settings within the ORCA software were adapted to the system's needs and synchronization with the FPD was set up.
In the commissioning phase for the muon detection system, different tests were performed to ensure a smoothly working system. The single PMTs were tested with a Sr source revealing two sides showing lower rates than the rest. This was compensated for by raising acceleration voltages for the affected sides. The stability of the system was investigated. It was found that natural atmospheric fluctuations cause much larger rate fluctuations than the module electronics. The efficiency of the single modules was examined and found to be \SI{93.4 \pm 3.4}{\percent}. The module's rates compare very well to literature values. 

It was shown that the muon induced electron rate is well shielded by axially symmetric magnetic fields and that, under different conditions, this rate increases strongly. This proved that the great efforts invested to achieve accurate field knowledge and settings are necessary and will be rewarded with low background measurements.
Analysis with both asymmetric and non axially symmetric fields were very successful showing that all induced events show similar times of flight from the vessel wall to the detector. At the main spectrometer, the setup still needs to be optimized. Due to the limited measurement time in the now ended SDS commissioning measurement phase, further Investigation of the Problem was not possible but will be in the future.



