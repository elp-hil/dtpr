%% appendix.tex
%%

%% ==============================
%\chapter{Appendix}
%\label{ch:Appendix}
%% ==============================

\appendix

\iflanguage{english}
{\addchap{Appendix}}	% english style
{\addchap{Anhang}}	% german style


\section{First Appendix Section}
		\label{appendix}
		
\setcounter{figure}{0}
		
\begin{figure} [ht]
  \centering
   ein Bild
  \caption{A figure}
  \label{fig:BPMNBeispiela}
\end{figure}


\begin{figure} [ht]
  \centering
    %\includegraphics{graphics/ORCA/script}
  \caption[ORCA script LFCS coils]{Scripting task through the example of a LFCS current ramping script. All currents are incremented in thenths of the maximum current for the individual coil. }
  \label{fig:ORCA:script}
\end{figure}



\begin{table}
			\begin{tabular}{C{\textwidth}}
			{\bf non-axially symmetric magnetic field}\\
		\end{tabular}\\
		\begin{tabularx}{\textwidth}{|>{\centering}X>{\centering}X>{\centering}X>{\centering}X>{\centering}X>{\centering}X>{\centering\arraybackslash}X|}
			\hline
			\centering
			solenoid source &solenoid detector &inner aircoil & outer aircoil &outer cent. aircoil &emcs x	&emcs y\\
			\hline
			25	&25	&7	&-7	&5	&0	&-14\\
			\hline
		\end{tabularx}
		\begin{tabularx}{\textwidth}{|l|X|}
		\hline
			mos00159753- & Two horizontal loops at \SI{100}{\ampere}\\
			mos00159754 &\\
			\hline
			mos00159755-& Two horizontal loops at \SI{-100}{\ampere}\\
			mos00159758 &\\
			\hline
			mos00159759-& No current in horizontal loops - background measurement\\
			mos00159771 &\\
			\hline
			mos00159772-& Two horizontal loops at \SI{100}{\ampere}\\
			mos00159773 &\\
			\hline
		\end{tabularx}
		\vspace{0.5cm}
		
		\begin{tabularx}{\textwidth}{|>{\centering}X>{\centering}X>{\centering}X>{\centering}X>{\centering}X>{\centering}X>{\centering\arraybackslash}X|}
			\hline
			\centering
			solenoid source &solenoid detector &inner aircoil & outer aircoil &outer cent. aircoil &emcs x	&emcs y\\
			\hline
			12.5	&12.5	&3.5	&-3.5	&2.5	&0	&0\\
			\hline
		\end{tabularx}
		\begin{tabularx}{\textwidth}{|l|X|}
			\hline
			mos00160661- & Two horizontal loops at \SI{50}{\ampere}\\
			mos00160666 &\\
			\hline
			mos00160667-&No current in horizontal loops - background measurement\\
			mos00160682 &\\
			\hline
			mos00160684-& Two horizontal loops at \SI{-50}{\ampere}\\
			mos00160687 &\\
			\hline
		\end{tabularx}
		\vspace{0.5cm}
		
		\begin{tabularx}{\textwidth}{|>{\centering}X>{\centering}X>{\centering}X>{\centering}X>{\centering}X>{\centering}X>{\centering\arraybackslash}X|}
			\hline
			\centering
			solenoid source &solenoid detector &inner aircoil & outer aircoil &outer cent. aircoil &emcs x	&emcs y\\
			\hline
			6.2 & 6.2 & 1.7 & -1.7 & 1.2 & 0 & 0\\
			\hline
		\end{tabularx}
				\begin{tabularx}{\textwidth}{|l|X|}
			\hline
			mos00160688- & Two horizontal loops at \SI{25}{\ampere}\\
			mos00160691 &\\
			\hline
			mos00160692-&No current in horizontal loops - background measurement\\
			mos00160706 &\\
			\hline
			mos00160707-& Two horizontal loops at \SI{-25}{\ampere}\\
			mos00160711 &\\
			\hline
		\end{tabularx}
		
\end{table}

\begin{figure}
\centering
	\includegraphics[width = 0.9\textwidth]{graphics/monSpec/nonAxiallySym/{+50_12.5_12.5}.eps}
	\caption[\SI{50}{\ampere} loops]{Two horizontal loops at \SI{50}{\ampere} current. For all settings see \ref{tab:analysis:nonAxiallySymmetricField}, line 1.}
	\label{fig:nonAxiallySym:+50_12.5_12.5.eps}
\end{figure}

\begin{figure}
\centering
	\includegraphics[width = 0.9\textwidth]{graphics/monSpec/nonAxiallySym/{-50_12.5_12.5}.eps}
	\caption[\SI{-50}{\ampere} loops]{Two horizontal loops at \SI{-50}{\ampere} current. For all settings see \ref{tab:analysis:nonAxiallySymmetricField}, line 3.}
	\label{fig:nonAxiallySym:-50_12.5_12.5.eps}
\end{figure}

\begin{figure}
\centering
	\includegraphics[width = 0.9\textwidth]{graphics/monSpec/nonAxiallySym/{+25_12.5_12.5}.eps}
	\caption[\SI{25}{\ampere} loops]{Two horizontal loops at \SI{25}{\ampere} current. For all settings see \ref{tab:analysis:nonAxiallySymmetricField}, line 2.}
	\label{fig:nonAxiallySym:+25_12.5_12.5.eps}
\end{figure}
\begin{figure}
\centering
	\includegraphics[width = 0.9\textwidth]{graphics/monSpec/nonAxiallySym/{-25_12.5_12.5}.eps}
	\caption[\SI{-25}{\ampere} loops]{Two horizontal loops at \SI{-25}{\ampere} current. For all settings see \ref{tab:analysis:nonAxiallySymmetricField}, line 4.}
	\label{fig:nonAxiallySym:-25_12.5_12.5.eps}
\end{figure}
\begin{figure}
\centering
	\includegraphics[width = 0.9\textwidth]{graphics/monSpec/nonAxiallySym/{BG_12.5_12.5}.eps}
	\caption[\SI{50}{\ampere} loops]{Two horizontal loops at \SI{0}{\ampere} current. For all settings see \ref{tab:analysis:nonAxiallySymmetricField}, line 1.}
	\label{fig:nonAxiallySym:BG_12.5_12.5.eps}
\end{figure}

\begin{figure}
\centering
	\includegraphics[width = 0.9\textwidth]{graphics/monSpec/nonAxiallySym/{BG_25_25}.eps}
	\caption[\SI{50}{\ampere} loops]{Two horizontal loops at \SI{0}{\ampere} current. Both solenoids at \SI{25}{\ampere} for a comparison of the background at different field widening. For all settings see \ref{tab:analysis:nonAxiallySymmetricField}, line 1.}
	\label{fig:nonAxiallySym:BG_25_25.eps}
\end{figure}


\begin{table}
\centering
	\begin{tabular}{| l | cc | cc| cc |}
	\hline
	Date & T$_{low}$ [K] & T$_{high}$ [K] & p$_{low}$ [kPa] & p$_{high}$ [kPa] & p$_l$ / T$_l$ & p$_h$ / T $_h$ \\
	\hline
	21.12.12 & 274.95 & 281.25 & 1010.10 & 1018.20 & 3.67 & 3.62\\
	22.12.12 & 278.55 & 282.15 & 1009.50 & 1020.60 & 3.62 & 3.62\\
	23.12.12 & 282.85 & 287.25 & 1009.50 & 1013.70 & 3.57 & 3.53\\
	24.12.12 & 277.05 & 287.15 & 1007.40 & 1013.50 & 3.64 & 3.53\\
	25.12.12 & 276.05 & 288.35 & 1004.00 & 1010.30 & 3.64 & 3.50\\
	26.12.12 & 281.25 & 282.85 & 1010.40 & 1016.40 & 3.59 & 3.59\\
	27.12.12 & 280.75 & 283.25 & 1004.80 & 1014.70 & 3.58 & 3.58\\
	28.12.12 & 279.65 & 281.85 & 1016.20 & 1029.50 & 3.63 & 3.65\\
	29.12.12 & 276.05 & 284.55 & 1014.90 & 1026.00 & 3.68 & 3.61\\
	30.12.12 & 279.05 & 282.85 & 1015.90 & 1024.60 & 3.64 & 3.62\\
	31.12.12 & 277.05 & 283.15 & 1011.60 & 1024.40 & 3.65 & 3.62\\
	01.01.13 & 274.45 & 281.45 & 1008.10 & 1016.90 & 3.67 & 3.61\\
	02.01.13 & 272.25 & 279.15 & 1017.50 & 1033.00 & 3.74 & 3.70\\
	03.01.13 & 273.65 & 280.45 & 1033.10 & 1038.30 & 3.78 & 3.70\\
	\hline
	
	\end{tabular}
	\caption[Temperature and pressure Rheinstetten]{Temperature and pressure data from the weather station in Rheinstetten. Daily high and low were given, included are the ratio of pressure and temperature for both the high and the low values. This ratio is proportional to the air's density . Bare in mind that this data is only for the low atmospheric layer and the station is also around \SI{20}{\kilo\meter} away  from the KATRIN muon modules.}
	\label{fig:weatherData}
\end{table}


\dots



