Die drei Neutrinos, nach der Postulation durch Pauli inzischen etabliertes Elementarteilchen, sind die einzigen Teilchen des Standardmodells, deren Masse bisher unbekannt ist. Zahlreiche Oszillationsexperimente haben gezeigt, dass die Masse nicht null sein kann, finden jedoch nur Zugang zu den Differenzen der Massenquadrate. Die Bestimmung einer dieser Massen, die des Elektron-Neutrino, hat sich das KATRIN Experiment ({Ka}rlsruher {Tri}tium {N}eutrino Experiment) zum Ziel gesetzt. Dabei nutzt es eine, im Gegensatz zu neutrinolosem doppeltem Beta-Zerfall oder kosmologischen Betrachtungen, modellunabh\"angige Methode. Die Zerfallselektronen des Tritium werden mit Hilfe eines MAC-E Filters 

Das Myon detektions System, das in dieser Arbeit beschrieben wird, ist Teil des KATRIN Experiments.