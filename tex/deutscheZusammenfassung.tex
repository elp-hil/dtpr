\chapter*{Deutschsprachige Zusammenfassung}

Die drei Neutrinos, nach der Postulation durch Pauli inzwischen etablierte Elementarteilchen, sind die einzigen Teilchen des Standardmodells, deren Masse bisher unbekannt ist. Zahlreiche Oszillationsexperimente haben gezeigt, dass die Masse endlich ist, finden jedoch nur Zugang zu den Differenzen der Massenquadrate. Die Bestimmung einer dieser Massen, die des Elektron-Neutrino, hat sich das KATRIN Experiment ({Ka}rlsruher {Tri}tium {N}eutrino Experiment) zum Ziel gesetzt. Dabei nutzt es eine, im Gegensatz zu neutrinolosem doppeltem Beta-Zerfall oder kosmologischen Betrachtungen, modellunabh\"angige Methode. Die Zerfallselektronen des Tritium werden mit Hilfe eines MAC-E Filters analysiert. Dieser parallelisert die Impulse der Elektronen aus einer isotrop strahlenden Quelle, um sie dann durch ein elektromagnetischen Potential zu analysiseren. Dazu ist ein raffiniertes System aus Supraleitern und normalleitenden Spulen n\"otig, die die Zerfallselektronen in einem magnetischen Flusschlauch von der Quelle zu einem Detektor f\"uhren. Betrachtet wird der Endpunkt des Spektrums, dessen Form von der Masse des Elektron-Neutrinos am st\"arksten beeinflusst wird. Das KATRIN Experiment wird diesen Endpunkt mit bisher unerreichter Genauigkeit darstellen. Es wird in der Lage sein, eine Neutrinomasse von \SI{0.2}{\electronvolt}/$c^2$ bei \SI{90}{\percent} C.L. zu messen und damit die Vorg\"angerexperimente von Mainz und Troisk um eine Gr\"ossenordnung \"ubertreffen. Um au\ss erdem den Betrag des elektrischen Potentials genau verfolgen zu k\"onnen, vermisst das Monitorspektrometer, an welchem im Rahmen dieser Arbeit ebenfalls Messungen durchgef\"uhrt wurden, Transmissionsfunktionen von $\beta$-Quellen bekannte Energie. 
F\"ur einen solch pr\"azisen Messaufbau sind die genaue Kenntnis aller Untergrundprozesse, die das Messergebnis verf\"alschen k\"onnen, notwendig. Neben Elektronen aus Zerf\"allen im Innern der Messaparatur k\"onnen solche auch extern induziert werden. Den relevanten Beitrag liefern hierbei Myonen aus kosmischen Luftschauern, die durch Streuung an den W\"anden des Tanks Elektronen aus diesed ausl\"osen. Als Gegenma\ss nahme sind im Innern des Spektrometertanks Elektroden installiert. Diese liegen auf einem negativeren Potenzial als die Tankwand und schirmen so den magnetischen Flusscshlauch im Innern gegen die Untergrund-Elektronen ab. Hochenergetische Elektronen k\"onnen jedoch noch immer in das sensitive Volumen eindringen. Ausserdem bieten die Elektroden selbst sowie ihre Haltestrukturen wiederum eine, wenngleich weitaus kleinere, Angriffsfl\"ache f\"ur Myonen.\\

Diese Arbeit besch\"aftigt sich mit dem Nachweis und der Simulation kosmischer Myonen sowie der Analyse der gewonnenen Daten.
Dazu wurde in ihrem Rahmen zun\"achst das aus acht Szintillatormodulen und Ausleseelektronik bestehende Myon Detektionssystem des Hauptspektrometers fertiggestellt. Der gesamte Aufbau wurde strukturiert verkabelt, Hochspannungsger\"ate wurden beschafft und installiert, die Erdung angebracht, Synchronisation mit dem Datennahmesystem des Detektors hergestellt.
Parallel wurden erste Analysen vorl\"aufiger Messungen zur Inbetriebnahme durchgef\"uhrt. Diese dienten weitestgehend dem Test und der Inbetriebnahme der Module. So wurden passende Software Gains und Thresholds gesetzt und Beschleunigungsspannungen eingestellt. Beim Test der Photmultiplier mithilfe einer Stronzium Quelle wurde die Notwendigkeit der Erh\"ohung der Beschleunigungsspannungen zweier Modulseiten festgestellt. Um die kurzen Pulse (\SI{20}{\nano\second}) der Module zuverl\"assig zu identifizieren, wurden verschiedene Software Filter mit einem Funktionengenerator auf ihre Energieaufl\"osung hin untersucht und optimiert.
Aufgrund der r\"aumlichen N\"ahe zum Low Field Compensation System (LFCS), das den magnetischen Flusschlauch formt, sind die Photomultiplier einem magnetischen Feld ausgesetzt, unter welchem sie nicht mehr zuverl\"assig funktionieren. Um dem entgegenzuwirken, wurden die Photomultiplier von einer Mu-Metallschicht umh\"ullt, die das Feld im innern aufgrund ihrer hohen magnetischen Permeabilit\"at verringert. Das neue Setup wurde unter den h\"ochsten zu erwartenden Feldern getestet und zeigte einen deutlich geringeren, akzeptablen Abfall der Rate.
Eine Langzeitmessung zeigte, dass die Module weit stabiler sind, als der durch nat\"urliche Fluktuationen der Atmosph\"are Temperatur beeinflusste Myonfluss. Die Effizienz der Module wurde zu \SI{93.4 \pm 3.4}{\percent} bestimmt.
Zur Verifizierung experimentell bestimmter Daten wurde eine Geant4-Simulation erstellt. In dieser wurden die Raten der Myonmodule verifiziert und sie kann weiter zur Simulation von Myon-induziertem Untergrund genutzt werden.\\\\
Die Messungen am bereits installierten Myon Detektionssystem am Monitorspektrometer wurden wieder aufgenommen. Dabei sind Messungen mit asymmetrischem und nicht axialsymmetrischem Feld durchgef\"uhrt worden. Bei asymmetrischem Feld verbinden magnetische Feldlinien die Wand des Spektrometers mit dem Detektor - die Myon induzierten Elektronen werden magnetisch zum Detektor gef\"uhrt. Bei nicht axialsymmetrischem Feld wird das Feld durch das Zuschalten einer Spule verformt, sodass die magnetische Reflexion, die den Untergrund sonst abschirmt, weniger gut wirkt. So k\"onnen Elektronen den Detektor durch $E\times B$ drifts mit weit h\"oherer Wahrscheinlichkeit erreichen. 
Zur Auswertung der Daten wurde auf das Myon Detektionssystem zugeschnittene Software geschrieben. Diese wurde f\"ur an Haupt- und Monitorspektrometer gewonnene Messungen genutzt.
In den Messungen konnte gezeigt werden, dass die Myon induzierte Elektronen Rate am Detektor, die \"uber zeitliche Korrelation zum Detektor-Event ($\approx$ \SI{1,5}{\micro\second} sp\"ater) identifiziert wurde, mit der Symmetrisierung des Feldes abnimmt. W\"ahrend der \glqq SDS comissioning measurements\grqq, einer ersten Messphase am Spektrometer und Detektor System wurden erste Untersuchungen des Myon induzierten Untergrundes am Hauptspektrometer durchgef\"uhrt. Diese Messungen zeigten trotz mehrfacher Anpassung des magnetischen Setups keine klaren zeitlichen Korrelationen zwischen Myon Detektionen und Detektor Events. Da in Simulationen gezeigt wurde, dass die Flugzeiten der Elektronen im Bereich der Rate der die vom Flussschlauch abgebildeten Fl\"ache durchdringenden Myonen liegt, sollten zuk\"unftige Messungen den Anspruch haben diese Fl\"ache so klein wie m\"oglich zu halten. Dazu bietet sich auch die Analyse einzelner Detektorpixel oder -ringe an, hier war die Analyse aufgrund der begrenzten Messdauer jedoch statistisch limitert.\\\\
Mit dem Abschluss dieser Arbeit wird ein voll funktionsf\"ahiges und intensiv getestetes System \"ubergeben, mit welchem gezeigt werden konnte, wie wichtig die Kenntnis aller Magnetfelder und ihre Symmetrisierung sind. Zudem wurden Softwarepakete zur Simulation und Auswertung gewonnener Daten erstellt, die zuk\"unftige Messungen erleichtern und Voraussagen zulassen.




