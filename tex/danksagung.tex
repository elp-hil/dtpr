\chapter*{Danksagung}
Bei der Entstehung dieser Arbeit hatte ich Unterst\"ustzung von zahlreichen Personen, ohne die ich die Vielzahl an Probleme und Aufgaben nicht in diesem Umfang h\"atte l\:osen k\"onnen. Daf\"ur bedanke ich mich sehr herzlich. Besonderer Dank geht an:

Prof. Drexlin daf\"ur, meine Diplomarbeit am Gro\ss projekt KATRIN schreiben zu d\"urfen und daf\"ur, durch eine interessante Vorlesung mein Interesse an Teilchenphysik geweckt zu haben.\\\\
Prof. Husemann f\"ur das Interesse an meiner Arbeit und die \"Ubernahme der Zweitkorrektur.\\\\
Nancy Wankowsky f\"ur die \"Ubernahme der Betreuung und die interessante Themenstellung sowie zahllose Korrekturen ohne die ich wahrscheinlich verzweifel w\"are \\\\
Benjamin Leiber f\"ur zahlreiche Hilfestellungen, das Korrekturlesen, eine gute Zusammenarbeit und Messphasen mit Wartezeitverk\"urzung bei Rad \& co.\\\\
Moritz Erhard und Norman Hau\ss mann f\"ur das mehrfach gek\"urte beste B\"uro, sehr hilfreiche Kommentare und Hilfestellungen w\"ahrend des Schreibens - sowie  f\"ur das "constant noise level" \cite{transformer}, das mir die Zeit bei KATRIN vers\"u\ss t hat.\\\\
Den Gaunern vom Detektor, die sich immer bem\"uht haben, dem kleinen aus dem Keller zu helfen - und dabei kein Blatt vor den Mund genommen haben.\\\\
Marco Haag f\"ur zahlreiche Hilfestellungen beim L\"osen von C++ Problemen, die f\"ur ihn keine waren.\\\\
Florian Fr\"ankle, der sich immer, wenn er am Projekt war, um meine Belange gek\"ummert hat.\\\\
Denis Tcherniakhovski f\"ur die grossartige Hilfe bei der Detektorsynchronisierung und anderen Fragen zur DAQ.\\\\
Arman Beglarian und Stefan G\"orhardt f\"ur die Hilfe bei der Kalibrierung der Luftspulen und die angenehme Arbeitsatmosph\"re.\\\\
\\
Und zuletzt meine Eltern, die mir das Studium erm\"glicht, mich aber bei weitem nicht nur finanziell unterst\"utzt haben, meinen Schwestern auf die ich immer z\"ahlen kann und Sinah - die wohl am meisten unter der Endphase gelitten hat, aber immer cool geblieben ist.


