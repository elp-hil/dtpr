%% simulationSoftware.tex
%%

%% ==============
\chapter{Simulation of muon induced background}
\label{ch:Simulation of muon induced background}
%% ==============
  To compare the data aquired to theoretically expected values, a Geant4 simulation of cosmic showers has been set up including the geometry of the main spectrometer as well as the muon modules. Using this software, any number of inciding muons can be simulated and the effect on the main spectrometer and the muon modules can be evaluated.
  %% ===========================
  \section{Geant4}
  \label{ch:Simulation software:sec:Geant4}
  %% ===========================
  The Geant4 package is a powerful tool for simulation of particles. It has loads of possible interactions already integrated making it easy for the user to set up and run simulations. After setting up geometry and detectors, the user starts a run. Each run may consist of one or more events. During a single run, a loop of processes is called:
  \begin{itemize}	
  	\item Primary Generator Action
  	\item Run action
  	\item Event action
  	\item Stacking action
  	\item Tracking action
  	\item Stepping action
  \end{itemize}
  For user interaction, for each item above, classes with the addition 'user' to the base classes name can be called before or after the standard action class. An example is the class G4UserEventAction invoked before and after each call of G4TrackingAction. It contains two member classes, namely BeginOfEventAction and EndOfEventAction that let the user decide what to do at this point. Through those, it is possible to change behaviour of the simulation or extract data needed. In this case, for every event, if more than one module has been hit, the copy numbers of those are pushed back to a vector of event data.  
  Running the simulation, one can either interactively enter commands or write those to a .mac file, by default the vis.mac file, which are then sequentially executed. 


  %% ===========================
  \section{Geometry setup}
  \label{ch:Simulation software:sec:Geometry setup}
  %% ===========================
  To set up a geometry, the class G4VUserDetectorConstruction is used. B1DetectorConstruction inherits from that as a base class and additionally contains all of the geometrical parameters needed for the setup such as radii of the main spectrometer cones or positions and extent of the muon modules. Every shape generated is made up of both a logical volume G4LogicalVolume and a physical volume G4PhysicalVolume. The logical volume describes the intrinsic properties of the geometric obect added: its shape, its size and its material. The physical volume accepts a logical volum as input providing position and alignment of the previously defined.
  Inside the detector construction class, all of the materials used in the simulation need to be defined as well. These are the components of the air outside and inside the spectometer including pressures and constitution, the stainless steel of the spectrometer wall and the scintillator material of the muon modules.
  The main spectrometer geometry was already \todo{written by who?} existant but had to be modified as many border volumes were implemented. These were very flat volumes covering any area of the main spectrometer not needed for this simulation. Additionally, the muon modules have been added as sensitive volume. Keeping in mind that one wants to not only distinguish wheter a module has been hit, but also which one. That is why the logical volume for every module is the same wheras the physical volume is a copy of the first at different world coordinates making them identifiable via their individual copy number.
  
  %% ===========================
  \section{Muon generator}
  \label{ch:Simulation software:sec:Muon generator}
  %% ===========================  
  
  Muon generation was realised through the primary generator action. The angular distribution suggested by Henrik Arlinghaus \todo{cite} was implemented. The angular rate dependance is shown in \todo{ref figure}.The energy was set to \todo{reasonable value} disregarding the actual anergy distribution as this was mainly about flight paths that are not strongly dependant on energy at high energies. Starting positions were spherically distributed, with the direction towards the origin, which is in the center of the main spectrometer. Positions were then randomly moved in a volume surrounding the spectrometer to account for the non-point like structure of the detection system as a whole, while the distribution describes a single point in space.
  
  %% ===========================
  \section{Hit counter}
  \label{ch:Simulation software:sec:Hit counter}
  %% ===========================
  
  For comparing the simulation to real data, of all the events generated, those hitting modules were counted. Each event containing at least one hit is written into a file for further analysis. This made it possible to compare the rates of the single modules, showning that the generator works fine. furthermore, it allowed for a estimation of muons hitting the modules compared to the total of inciding muons.
  