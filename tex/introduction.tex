%% introduction.tex
%%

%% ==============================
\chapter{Introduction}
\label{ch:Introduction}
%% ==============================
    \section{Neutrinos in the standard model}
    \label{ch:Introduction:sec:Neutrinos in the standard model}
    During the second part of the 20th century, a model has been developed to describe a huge portion of phenomena stating 16 particle, that is six quarks, six leptons (both made up of three particle generations) and the four Gauge Bosons. The latter are carriers of the standard models interactions of the former particles, meaning all interactions of matter are based on the exchange of one or more of the Gauge Bosons. In the standard model, neutrinos are considered massless. 
    Many experiments have shown that the weightless neutrino is a wrong assumption. Most of these were experiments prooving neutrino oscillatinos with both reactor neutrinos and solar neutrinos such as \todo{add experiments}.
    
    \subsection{Neutrino Oscillations}
    \label{ch:Introduction:sec:Massive neutrino:subsec:neutrino Oscillations}
      If the neutrinos were without mass, its mass eigenstates would equal its flavour eigenstates:
      
    \begin{equation}
	\begin{array}{ccc}
      	|\nu_e>		& = & |\nu_{m,e}>\\
      	|\nu_\mu>	& = & |\nu_{m,\nu}>\\
      	|\nu_\tau>	& = & |\nu_{m,\tau}>\\
    	 \end{array}
    \end{equation}
    First doubts concerning this asumptions occured as inconsistencies between the measured and the calculated solar $\nu$-flux occured. As the counton $\nu_e$ was too low, the theory of neutrino oscillations emerged stating, that a mixture of flavours was possible as the flavours were made up of all three of the mass eigenstates. The Mixture is described by the so called Pontecorvo–Maki–Nakagawa–Sakata matrix:
        \begin{equation}
        \left(
        \begin{array}{c}
	  |\nu_e>\\
	  |\nu_\mu>\\
	  |\nu_\tau>\\
        \end{array}
        \right)
	 = \left(
	\begin{array}{ccc}
      	\theta_{e,e} & \theta_{e,\mu} & \theta_{e,\tau}\\
      	\theta_{\mu,e} & \theta_{\mu,\mu} & \theta_{\mu,\tau}\\
      	\theta_{\tau,e} & \theta_{\tau,\mu} & \theta_{\tau,\tau}\\
      	\end{array}
	\right)
	\left(
	\begin{array}{c}
      	|\nu_{m,e}>\\
      	|\nu_{m,\nu}>\\
      	|\nu_{m,\tau}>\\
    	 \end{array}
    	 \right)
    \end{equation}
    
     
    \subsection{Indirect measurement of neutrino mass}
    \label{ch:Introduction:sec:Massive neutrino:subsec:Measuring Neutrino Mass}

    \subsection{Direct measurement of neutrino mass}
    \label{ch:Introduction:sec:Massive neutrino:subsec:Measuring Neutrino Mass}
      
    \section{Cosmic muons and their interaction with matter}
    \label{ch:introduction:sec:Cosmic Air Showers}


